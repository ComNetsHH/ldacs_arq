\documentclass[a4paper]{article}
\usepackage{etex}
\usepackage[utf8]{inputenc}
\usepackage{lmodern}
\usepackage{hyphsubst}
\usepackage[english]{babel}
\usepackage{csquotes}
\usepackage{textcomp}
\usepackage{enumerate}
\usepackage{microtype}
\usepackage{listings}
\usepackage{graphicx}
\usepackage{subcaption}
%\usepackage{titling}
\usepackage{bbold}
\usepackage{float}
	\lstset{language=[LaTeX]TeX}
\usepackage[binary-units=true]{siunitx}
	\DeclareSIUnit{\belmilliwatt}{Bm}
	\DeclareSIUnit{\dBm}{\deci\belmilliwatt}
\usepackage{graphicx}
\usepackage{amsmath,amssymb,amsfonts}
\usepackage[nolist]{acronym}
\usepackage{tikz}
\usetikzlibrary{automata,positioning,fit}
\usepackage{algorithm}
\usepackage[noend]{algpseudocode}
\usepackage{todonotes}
\usepackage{bbold}
\usepackage[inline,shortlabels]{enumitem}
\usepackage{color}
\usepackage[hidelinks]{hyperref}
\usepackage[backend=biber, sorting=none, block=ragged]{biblatex}
\usepackage{adjustbox}
\renewcommand{\quote}[1]{``#1''}
\newcommand*\dif{\mathop{}\!\mathrm{d}} %correct spacing for integrals, usage: \int x \dif x
\newcommand{\round}[1]{\ensuremath{\lfloor#1\rceil}}
\begin{acronym}

\acro{4g}[4G]{Fourth Generation}
\acro{5g}[5G]{Fifth Generation}
\acro{6lowpan}[6LoWPAN]{IPv6 over Low-Power Wireless Personal Area Networks}
\acro{6lowpan-nd}[6LoWPAN-ND]{IPv6 over Low-Power Wireless Personal Area Networks with Neighbor Discovery}
\acro{6p}[6P]{6top Protocol}
\acro{6tisch}[6TiSCH]{IPv6 over the TSCH mode of IEEE 802.15.4e}
\acro{6top}[6top]{6TiSCH Operation Sublayer}
\acro{a2a}[A2A]{Air-to-Air}
\acro{a2g}[A2G]{Air-to-Ground}
\acro{aanet}[AANET]{Aeronautical Ad-hoc Network}
\acro{ac}[AC]{aircraft}
\acro{ack}[ACK]{acknowledgement}
\acro{adsb}[ADS-B]{Automatic Dependent Surveillance - Broadcast}
\acro{ae}[AE]{Auto Encoder}
\acro{afdx}[AFDX]{Avionic Full Duplex}
\acro{afh}[AFH]{Adaptive Frequency Hopping Spread Spectrum}
\acro{ai}[AI]{Artificial Intelligence}
\acro{ais}[AIS]{Automatic Identification System}
\acro{ann}[ANN]{Artificial Neural Network}
\acro{aodv}[AODV]{Ad hoc On-Demand Distance Vector}
\acro{aodvv2}[AODVv2]{Ad Hoc On-demand Distance Vector Version 2}
\acro{ap}[AP]{Access Point}
\acro{arns}[ARNS]{Aeronautical Radionavigation Service}
\acro{arq}[ARQ]{Automatic Repeat Request}
\acro{asn}[ASN]{Absolute Slot Number}
\acro{atc}[ATC]{Air Traffic Control}
\acro{atm}[ATM]{Air Traffic Management}
\acro{awgn}[AWGN]{Additive White Gaussian Noise}
\acro{bc}[BC]{Broadcast Channel}
\acro{ber}[BER]{Bit Error Rate}
\acro{bmwi}[BMWi]{Bundesministerium für Wirtschaft und Energie}
\acro{ban}[BAN]{Body Area Network}
\acro{cca}[CCA]{Clear Channel Assessment}
\acro{ccdf}[CCDF]{Complementary Cumulative Distribution Function}
\acro{cdf}[CDF]{Cumulative Distribution Function}
\acro{cnn}[CNN]{Convolutional Neural Network}
\acro{cr}[CR]{Cognitive Radio}
\acro{crn}[CRN]{Cognitive Radio Network}
\acro{csmaca}[CSMA/CA]{Carrier Sense Multiple Access/Collision Avoidance}
\acro{csma}[CSMA]{Carrier Sense Multiple Access}
\acro{cstdma}[CSTDMA]{Carrier Sense TDMA}
\acro{css}[CSS]{Cooperative Spectrum Sensing}
\acro{darpa}[DARPA]{U.S. Defense Advanced Research Projects Agency}
\acro{d2d}[D2D]{Device-to-Device}
\acro{dbn}[DBN]{Deep Belief Network}
\acro{dl}[DL]{downlink}
\acro{dllayer}[DL]{data link}
\acro{dme}[DME]{Distance Measuring Equipment}
\acro{dnn}[DNN]{Deep Neural Network}
\acro{dodag}[DODAG]{Destination Oriented Directed Acyclic Graph}
\acro{drl}[DRL]{Deep Reinforcement Learning}
\acro{dsme}[DSME]{Deterministic and Synchronous Multi-channel Extension}
\acro{dsss}[DSSS]{Direct-Sequence Spread Spectrum}
\acro{dtmc}[DTMC]{discrete-time Markov chain}
\acro{dtx}[DTX]{discontinuous reception}
\acro{dqn}[DQN]{Deep $Q$-Network}
\acro{eb}[EB]{Enhanced Beacon}
\acro{ed}[ED]{Energy Detection}
\acro{embb}[eMBB]{Enhanced Mobile Broadband}
\acro{enb}[eNB]{evolved NodeB}
\acro{etx}[ETX]{Expected Transmission Count}
\acro{fanet}[FANET]{Flying ad hoc Network}
\acro{fdma}[FDMA]{Frequency-Division Multiple Access}
\acro{fdr}[FDR]{Frame Delivery Ratio}
\acro{fec}[FEC]{Forward Error Correction}
\acro{fft}[FFT]{Fast Fourier Transformation}
\acro{fhss}[FHSS]{Frequency Hopping Spread Spectrum}
\acro{fl}[FL]{Flight Level}
\acro{flink}[FL]{Forward Link}
\acro{fmcw}[FMCW]{Frequency Modulated Continuous Wave}
\acro{frr}[FRR]{frequency resource reuse}
\acro{fspl}[FSPL]{Free-Space Path Loss}
\acro{gan}[GAN]{Generative Adversarial Network}
\acro{gd}[GD]{Gradient Descent}
\acro{ge}[GE]{Gilbert-Elliot}
\acro{gps}[GPS]{Global Positioning System}
\acro{harq}[HARQ]{Hybrid Automatic Repeat Request}
\acro{hmm}[HMM]{Hidden Markov Model}
\acro{hs}[HS]{Hopping Sequence}
\acro{icao}[ICAO]{International Civil Aviation Organization}
\acro{idm}[IDM]{Interference Detection Mechanism}
\acro{ieee}[IEEE]{Institute of Electrical and Electronics Engineers}
\acro{ie}[IE]{Information Element}
\acro{iesg}[IESG]{Internet Engineering Steering Group}
\acro{ietf}[IETF]{Internet Engineering Task Force}
\acro{intairnet}[IntAirNet]{Inter Aircraft Network}
\acro{ioe}[IoE]{Internet of Everything}
\acro{iot}[IoT]{Internet of Things}
\acro{iiot}[IIoT]{Industrial Internet of Things}
\acro{iov}[IoV]{Internet of Vehicles}
\acro{ip}[IP]{Internet Protocol}
\acro{ipm}[IPM]{Interference Prediction Mechanism}
\acro{irtf}[IRTF]{Internet Research Task Force}
\acro{ism}[ISM]{industrial, scientific and medical}
\acro{itu}[ITU]{International Telecommunication Union}
\acro{itur}[ITU-R]{International Telecommunication Union Radiocommunication Sector}
\acro{jtids}[JTIDS]{Joint Tactical Information Distribution System}
\acro{kpi}[KPI]{Key Performance Indicator}
\acro{ldacs}[LDACS]{L-band Digital Aeronautical Communications System}
\acro{ldg}[LDG]{Landing Gear}
\acro{lln}[LLN]{Low-power and Lossy Network}
\acro{lorawan}[LoRaWAN]{Long Range Wide Area Network}
\acro{lpwan}[LPWAN]{Low-Power Wide Area Network}
\acro{lqi}[LQI]{Link Quality Indicator}
\acro{lre}[LRE]{Limited Relative Error}
\acro{lstm}[LSTM]{Long Short Term Memory}
\acro{lte}[LTE]{Long Term Evolution}
\acro{lte-u}[LTE-U]{Long Term Evolution Unlicensed}
\acro{lufo}[LuFo]{Luftfahrtforschung}
\acro{mac}[MAC]{Medium Access Control}
\acro{amanet}[avionic MANET]{Avionic Mobile Ad-hoc Network}
\acro{manet}[MANET]{Mobile Ad-hoc Network}
\acro{mcs}[MCS]{Modulation and Coding Scheme}
\acro{mcsotdma}[MC-SOTDMA]{Multi-Channel Self-Organized Time Division Multiple Access}
\acro{mdp}[MDP]{Markov Decision Process}
\acro{ml}[ML]{Machine Learning}
\acro{mlp}[MLP]{Multi-Layer Perceptron}
\acro{mmtc}[mMTC]{massive Machine-Type Communications}
\acro{mos}[MOS]{Mean Opinion Score}
\acro{msfg}[MSFG]{Matrix Signal Flow Graph}
\acro{msf}[MSF]{Minimal Scheduling Function}
\acro{msk}[MSK]{Minimum-Shift Keying}
\acro{mso}[MSO]{Message Start Opportunity}
\acro{mtsch}[MTSCH]{Mobility-aware Time Slotted Channel Hopping}
\acro{mtu}[MTU]{Maximum Transmission Unit}
\acro{nack}[NACK]{negative acknowledgment}
\acro{nac}[NAC]{North Atlantic Corridor}
\acro{nato}[NATO]{North Atlantic Treaty Organization}
\acro{nbiot}[NB-IoT]{Narrowband Internet of Things}
\acro{nextgen}[NextGen]{Next Generation Air Transportation System}
\acro{nice}[NICE]{Non-Intrusive Channel quality Estimation}
\acro{ofdm}[OFDM]{Orthogonal Frequency Division Multiplexing}
\acro{olsr}[OLSR]{Open Link State Routing}
\acro{oqpsk}[OQPSK]{Offset Quadrature Phase-Shift Keying}
\acro{osi}[OSI]{Open Systems Interconnection}
\acro{p2p}[P2P]{Point-to-Point Channel}
\acro{pan}[PAN]{Personal Area Network}
\acro{pca}[PCA]{Priority Channel Access}
\acro{pdf}[PDF]{Probability Density Function}
\acro{pdr}[PDR]{Packet Delivery Ratio}
\acro{pdu}[PDU]{Protocol Data Unit}
\acro{per}[PER]{Packet Error Rate}
\acro{pf}[PF]{Proportional Fair}
\acro{pgf}[PGF]{Probability Generating Function}
\acro{phy}[PHY]{Physical}
\acro{pib}[PIB]{PAN Information Base}
\acro{pl}[PL]{Packet Loss}
\acro{pmf}[PMF]{Probability Mass Function}
\acro{ppp}[PPP]{Poisson Point Process}
\acro{prb}[PRB]{Physical Resource Block}
\acro{prr}[PRR]{Packet Reception Rate}
\acro{psd}[PSD]{Power Spectral Density}
\acro{pu}[PU]{Primary User}
\acro{qos}[QoS]{Quality of Service}
\acro{ra}[RA]{Radio Altimeter}
\acro{rbm}[RBM]{Restricted Boltzmann Machine}
\acro{rb}[RB]{resource block}
\acro{rch}[RCH]{Random Channel Hopping}
\acro{rd}[RD]{Relative Direction}
\acro{relu}[ReLu]{Rectified Linear Unit}
\acro{rfc}[RFC]{Request for Comments}
\acro{rlc}[RLC]{Radio Link Control}
\acro{rl}[RL]{Reinforcement Learning}
\acro{rlink}[RL]{Reverse Link}
\acro{rlnc}[RLNC]{Random Linear Network Coding}
\acro{rnn}[RNN]{Recurrent Neural Network}
\acro{rpl}[RPL]{IPv6 Routing Protocol for Low-Power and Lossy Networks}
\acro{rr}[RR]{Round Robin} 
\acro{rssi}[RSSI]{Received Signal Strength Indicator}
\acro{rsvp}[RSVP]{Resource Reservation Protocol}
\acro{rt}[RT]{real~time}
\acro{rtt}[RTT]{Round Trip Time}
\acro{rtx}[RTX]{Retransmit}
\acro{rx}[RX]{Receive}
\acro{sdu}[SDU]{Service Data Unit}
\acro{sesar}[SESAR]{Single European Sky ATM Research}
\acro{srej}[SREJ]{Selective Rejection}
\acro{sf0}[SF0]{Scheduling Function 0}
\acro{sf1}[SF1]{Scheduling Function 1}
\acro{sfg}[SFG]{Signal Flow Graph}
\acro{sfl}[SFL]{Slot Frame Length}
\acro{sfsb}[SFSB]{Scheduling Function with Soft Blacklisting}
\acro{sf}[SF]{Scheduling Function}
\acro{sfx}[SFX]{6TiSCH Experimental Scheduling Function}
\acro{sinr}[SINR]{Signal-to-Interference-plus-Noise Ratio}
\acro{snr}[SNR]{Signal-to-Noise Ratio}
\acro{sotdma}[SOTDMA]{Self-organized TDMA}
\acro{sr}[SR]{Speed Ratio}
\acro{ssr}[SSR]{Secondary Surveillance Radar}
\acro{su}[SU]{Secondary User}
\acro{svm}[SVM]{Support Vector Machine}
\acro{sw}[SW]{Stop-and-Wait}
\acro{tdma}[TDMA]{Time-Division Multiple Access}
\acro{tg}[TG]{Task Group}
\acro{tsch}[TSCH]{Time Slotted Channel Hopping}
\acro{tsn}[TSN]{Time-Sensitive Networking}
\acro{ts}[TS]{Time Slot}
\acro{tti}[TTI]{transmission time interval}
\acro{ttl}[TTL]{Time To Live}
\acro{tx}[TX]{Transmit}
\acro{uat}[UAT]{Universal Access Transceiver}
\acro{uav}[UAV]{unmanned aerial vehicle}
\acro{ue}[UE]{User Equipment}
\acro{ul}[UL]{uplink}
\acro{urllc}[URLLC]{Ultra-Reliable Low-Latency Communications}
\acro{v2i}[V2I]{Vehicle-to-Infrastructure}
\acro{v2v}[V2V]{Vehicle-to-Vehicle}
\acro{vae}[VAE]{Variational Auto Encoder}
\acro{vdl}[VDL]{VHF Data Link}
\acro{vhf}[VHF]{Very high frequency}
\acro{voip}[VoIP]{Voice-over-IP}
\acro{w3c}[W3C]{World Wide Web Consortium}
\acro{waic}[WAIC]{Wireless Avionics Intra-Communication}
\acro{wg}[WG]{Working Group}
\acro{wifi}[WiFi]{IEEE 802.11 family of wireless technologies}
\acro{wlan}[WLAN]{Wireless Local Area Network}
\acro{wlog}[WLOG]{without loss of generality}
\acro{wrc}[WRC]{World Radiocommunication Conference}
\acro{wsn}[WSN]{Wireless Sensor Network}

\end{acronym}
\usepackage{float}
\usepackage{rotating}
\usepackage{bm}
\usetikzlibrary{decorations.pathreplacing}

\addbibresource{IntAirNet.bib}

\renewcommand{\quote}[1]{``#1''}

\makeatletter
\renewcommand{\maketitle}{
	\pagenumbering{gobble}
	\begin{minipage}{0.49\textwidth}
		\begin{flushleft}\includegraphics[width=0.9\textwidth]{imgs/logos/logo_comnets.png}\end{flushleft}
	\end{minipage}	
	\begin{minipage}{0.49\textwidth}
		\begin{flushright}\includegraphics[width=0.8\textwidth]{imgs/logos/logo_tuhh.png}\end{flushright}
	\end{minipage}	
	\vspace*{1em}
	\begin{center}
		~\\[2em]
		{\huge{}Specification}
		\\[2em]
		{\Huge{}\@title}
		\\[3em]
		{\huge{}Inter Aircraft Network}
		\vfill
		
		% replace with correct logo
		\includegraphics[width=0.4\textwidth]{imgs/logos/bmwi.png}

		{\Large\@date}
		\vspace*{5em}
		
		\begin{minipage}[h][2em][b]{0.49\textwidth}
			\begin{flushleft}
				Förderkennzeichen: 20V1708F
			\end{flushleft}			
		\end{minipage}
		\begin{minipage}[h][2em][b]{0.49\textwidth}
			\begin{flushright}
				Project Duration:\\01/2019 -- 03/2022
			\end{flushright}			
		\end{minipage}
		\noindent\makebox[\linewidth]{\rule{\textwidth}{0.4pt}}
		\begin{minipage}[h][6em][b]{0.44\textwidth}
			\begin{flushleft}
				\textbf{Work Package (WP) Leader:}\\Prof. Dr.-Ing. Andreas Timm-Giel
				\\
				\textbf{Partners:}\\DLR, Rohde \& Schwarz, NavCert, f.u.n.k.e., BPS, iAd
			\end{flushleft}
		\end{minipage}
		\begin{minipage}[h][6em][b]{0.54\textwidth}
			\begin{flushright}
				\textbf{Contributors:}\\\mbox{Sebastian Lindner, M.Sc.}\\\mbox{Konrad Fuger, M.Sc.}\\
\mbox{}\\
\mbox{}
			\end{flushright}
		\end{minipage}
	\end{center}
	\newpage
}
\hypersetup{
	pdftitle=TUHH Inter Aircraft Network Deliverable AP 2.2,
	pdfauthor=Hamburg University of Technology (TUHH)
}

\newcommand{\nr}{\texttt{seqno\_{}rx}}
\newcommand{\ns}{\texttt{seqno\_{}tx}}
\newcommand{\poll}{\texttt{poll}}
\newcommand{\srej}{\texttt{srej}}
\newcommand{\srejnum}{\texttt{srej\_{}num}}

\newcommand{\vr}{\texttt{V\_{}seqno\_{}rx}}
\newcommand{\vs}{\texttt{V\_{}seqno\_{}tx}}
\newcommand{\va}{\texttt{V\_{}seqno\_{}unack}}
\newcommand{\pr}{\texttt{V\_{}poll\_{}required}}
\newcommand{\srejr}{\texttt{V\_{}srej\_{}request\_{}list}}
\newcommand{\srejw}{\texttt{V\_{}srej\_{}waiting\_{}list}}
\newcommand{\rtx}{\texttt{V\_{}rtx\_{}list}}

\begin{document}
	\title{\quote{Avionic Automatic Repeat Request Protocol}}
	\date{October 2020}
	\maketitle
	
	\tableofcontents
	\newpage
	\pagenumbering{arabic}
	
	\section{Introduction}
	To establish reliable communication between IntAirNet hosts, an \ac{arq} protocol is used. 
	The following provides a specification od the protocol as well as the documentation of the implementation in the simulator.

	\subsection{Description}
	To obtain maximal channel utilization, the IntAirNet \ac{arq} is a version of selective repeat \ac{arq} and therefore aims to only retransmit segments which are explicitly not received.
	To do so, the \ac{arq} protocol uses cumulative \acp{ack} as well a selective rejection list to indicate \ac{nack} of segments which have not been received.

	\subsection{Architecture}
	The \ac{arq} is a link layer sublayer which resides between the \ac{mac} and the \ac{rlc} layer.
	Every transmission is initiated by the \ac{mac} layer.
	For this purpose the \ac{mac} layer monitors the buffer status of both \ac{rlc} and \ac{arq} layer and establishes transmission oppurtunities accordingly.
	Whenever a transmssion oppurtunity arrives, the \ac{mac} layer will request a segment from \ac{arq} layer which will either return an unacknowledged segment for retransmission or in turn request a new segment from the \ac{rlc} layer.


	\subsection{Header Fields}\label{sec:arq_header}
		The following header fields are required for the operation of the \ac{arq} protocol.		
		
		\begin{description}
			\item[\nr{}] The \quote{next sequence number to receive} field \nr{} carries the \emph{next} expected sequence number at the sender of this message.
			This acknowledges the reception of all sequence numbers up to $\nr{}-1$.
			
			\item[\ns{}] The transmit sequence number is the sequence number of the current segment that is being transmitted.
			
			\item[\poll{}] The poll field is set to an offset in the number of slots reserved for this communication link.
			The slot denotes a moment in time at which an \ac{ack} is requested at the latest.
			If a reverse link is configured, the remote end will piggyback its \acp{ack} onto each transmission.
			The slot denoted by the field is updated every time a segment is received.
			If \acp{ack} continue to arrive on the reverse link, then the respective slot will never be reached.
			If it is reached, then \acp{ack} stopped arriving or a reverse link is not configured or broke off.
			In this case, the respective slot is used for the remote end to send an \ac{ack}, and the local end abstains from transmission.	
			
			\item[\srejnum{}] The \texttt{srej\_{}num} field contains the number of entries in the \srej{} field.
			
			\item[] The \srej{} field contains as many sequence numbers as specified in the \texttt{srej\_{}num} field.
			Each sequence number is consequently \ac{nack}'d and should be scheduled for retransmission.
		\end{description}
		
	\subsection{System Variables}\label{sec:arq_variables}		
		The following system variables must be maintained at an \ac{arq} instance.			
		
		\begin{description}
			\item[\vs{}] The \vs{} variable contains the sequence number of the next in-order segment to be transmitted.
			
			\item[\vr{}] The \vr{} variable contains the sequence number of the next in-order segment expected to arrive.
			
			
			\item[\va{}] The \va{} variable contains the sequence number of the oldest un-acknowledged segment.
			
			\item[\pr{}] Number of slots when a \poll{} is required.
			
			\item[\srejr{}] List of segments that were detected as missing but not yet requested.
			
			\item[\srejw{}] List of segments that were detected as missing and requested but could not be \ac{ack}'d yet.
			
			\item[\rtx{}] List of segments that are currently required to be retransmitted.
		\end{description}
		
	\subsection{Sending data segments}\label{sec:arq_sending}
		Data segments are used to send new data or to retransmit lost segments.
		For new data the segment is assigned the sequence number from the \vs{} variable.
		The \vs{} variable is then incremented.
		For retransmitted data, the segment is assigned its original sequence number and the \vs{} is not affected.
		
		Generally, responses to the remote end take priority over other sending activities.
		This is to ensure that each \ac{pdu}'s packet delay is minimal, as application requirements might discard packets that take too long.
		Also, stalls are to be avoided when a sending window is exhausted.
		Therefore missing segments should be retransmitted and acknowledged as quickly as possible.
		
		Notifying the remote end of missing segments takes highest priority.
		If the \srejr{} list is not empty, then as many sequence numbers from it as fit are put into the \srej{} field and the number into the \srejnum{} field.
		These sequence numbers are then moved from the \srejr{} list into the \srejw{} list.
		The remaining capacity may be used for new data transmission.
		
		The next priority is to send outstanding retransmissions.
		Each item in the \rtx{} list is checked: if the number of attempted retransmissions exceeds the maximum allowed, then the segment is discarded.
		The current segment is given a header with a sequence number of the corresponding segment, but the data field is left empty.
		
		When new data is sent, the \poll{} field is filled with the current value of \pr{}.
		This variable is then updated to the end of the sending window minus $1$ ($\text{\va} + \frac{N}{2}-2$) unless it is already set to a smaller value.
		
	\subsection{Receiving data segments}\label{sec:arq_receiving}
		Upon reception, a segment's \ns{} header field is compared to the local \vr{} variable: $\vr{} - 1 + \frac{N}{2} - 1 \geq \ns{} \geq \vr{}$ should hold.
		Also the \nr{} field is checked for validity: $\vs{} \geq \nr{} \geq \va{}$ should hold.
		If either check fails, then the segment is discarded as invalid.
		
		If the segment's sequence number is identical to the next expected segment \vr{}, then it is passed into the reassembly buffer, and \vr{} is incremented.
		The current segment is removed if present from both the \srejr{} and \srejw{} lists.
		Next the reception buffer is checked for contiguous segments.
		All contiguous segments are added to the reassembly buffer and \vr{} is incremented accordingly.
		
		If the segment is not the next expected segment, then it is added to the reception buffer.
		A scheduled retransmission request is canceled by removing this segment from both the \srejr{} and \srejw{} lists.
		All sequence numbers between the highest previously received segment and the current segment are now deemed as missing and are added to the \srejw{} list, if not already on that list.
		
	\subsection{Poll time slot arrives}\label{sec:arq_poll}
		When the time slot at which a poll is requested arrives, then all sequence numbers from the \srejw{} list are transferred to the \srejr{} list, i.e. all missing segments are re-requested.
		All sequence numbers from the \srejr{} list are put into the \srej{} field, and \srejnum{} is set to the length of \srejr{}.
		The \nr{} and \ns{} fields are set to \vr{} and \vs{} as usual.
		This is an explicit \ac{ack}, so retransmission information has priority over user data.
		Remaining capacity may be filled with user data.		
			
	\appendix
	\section{References}
		\printbibliography[heading=none]
	
\end{document}